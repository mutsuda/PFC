%!TEX root = main.tex

The result of this Masther thesis is not a standalone application, but an enchancement of an already existing high performance and highly scalable application called FlowSight and developed by Zhilabs.
FlowSight captures and analyzes the traffic in a network and gives the final user the ability to see the data in many different ways. The higher level of interaction with FlowSight is performed through
a graphical user interface in a web browser. At a lower level, FlowSight is based on workflows, which are a series of configuration files that using the implemented modules are able to perform a task.



"FlowSight is a Massively Parallel Processing platform that scales using Commercial Off-The-Shelf Hardware and Directly Attached Storage."
Near the end of the year 2011 the company was asked by one of their Telecom clients to extend the functionality of one of the features their application was offering: they wanted to know not only which sites
where most visited by their subscribers, but also a generic categorization of those top sites. \\*
One way of solving this task would be doing it manually by someone in the company by reading those top sites and applying categories to them.
The problem of such a manual approach is that maybe today the client asked for the top 10 sites, but you never know when will they ask for the top 100000. 
Another way could be using the metadata information provided by the website itself, but not every site provides its category in the metadata.
The best approach to solve this task in a real scenario ended up being the artificial intelligence.
This project is the result of applying the power of machine learning algorithms to a real case scenario.

