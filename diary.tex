%!TEX root = main.tex

This annex contains the diary notes I took during the project development. It is written in catalan and it may contain partial or disconnected information. 
Reader discretion is advised.

\begin{itemize}
  \item Identificació del problema: On es determina la viabilitat de la construcció del SBC i la disponibilitat de les fonts de coneixement.
  \item Conceptualització: Descripció semiformal del coneixement del domini del problema i descomposició en subproblemes, segons la visió d'un expert.
  \item Formalització: Cal definir el mecanisme adequat de representació del coneixement, en aquest cas segons la visió de l'enginyer de coneixement.
  \item Rapidminer per testejar diferents solucions, naive bayes rapid, maxent lent pero + accuracy
  \item Categoritzacio URL per contingut -> MaxEnt
  \item Classificador basat en categories de Yahoo Directory (training)
  \item Testing inicial amb petites dades
  \item Testing posterior amb 140.000 URLs d'Irlanda O2
  \item Descartar idiomes estrangers -> Trigraphs
  \item Fitxers amb trigrams d'idiomes generats via randomly crawling wikipedia
  \item Crawling a traves de proxy amb headers tema idioma
  \item Afegida categoria adult
  \item Afegit categoria other, massa generica
  \item Desmembrat categoria other en altres per posterior mapeig a other
  \item Generat script per calcular mitja de certesa de classficacio i distancia mitja amb segona opcio
  \item Millorats fitxers de train en base al punt anterior
\end{itemize}


El primer approach per la classificació va ser fent servir les categories que demanava O2 irlanda. La categoria entertainment englobava molts temes, movies, tv, music,
literature..., i es requeria a més una categoria "Other". El classificador sempre assigna una categoria, la que té la màxima probabilitat de ser, i per tant s'havia 
de generar la categoria other a partir de categories que no tinguessin res a veure.
El problema de crear categoria generica other o entertainment, es que passaven a tenir molt ambigüitat, i moltes coses passaven a considerar-se other i entertainment.
La solucio va ser crear subcategories sense tenir en compte entertainment i other, i fer un post tractament del ficher resultant de la classificació assignant other o 
entertainment a allò que realment ho era. D'aquesta manera es té més granularitat, tot i que l'accuracy del classificador baixa al tenir més categories.
Un problema comú és que algunes urls no es deixen crawlejar. Detecten que no ets un navegador corrent i et donen un contingut que no és significatiu. A l'hora de
classificar aquestes urls acaben sent categoritzades a categories que no tenen res a veure. Per exemple wikipedia retorna simplement una llista de països, que fa que 
la categoria passi a ser "Adult". Coses similars passen amb facebook. 
A youtube el problema és que el text crawlejat són els títols i descripcions de videos que la gent puja, per tant depenent de quan es produeixi el crawling, la
classificacio de Youtube pot canviar.


\textbf{29-12-2011}

  En afegir categories concretes per a ser englobades posteriorment per Entertainment, l'accuracy del classificador ha baixat fins al 73.33
  adult       education  food\_drink  health             literature  music   real\_estate  religion  social\_networking  sport        travel
  automotive  email      games       history            maps        news    reference    science   social\_science     television   weather
  crime       financial  government  instant\_messaging  movies      photos  regional     shopping  software           theme\_parks

  Si la confusió és entre categories que seran englobades per "Entertainment" no hi ha problema. 
  Observem que la distància disminueix lògicament perquè ara hi ha més categories.
  \imatge{img/test.png}{Gràfic de blabla}

\textbf{04-01-2012}
 
  Proves a jabato
