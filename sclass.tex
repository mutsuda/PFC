%!TEX root = main.tex

The main goal of the project was to be able to automatically classify webpages depending on their content into different categories. 
By the goal itself it is clear that Machine Learning algorithms might be the best approach to try to solve the problem. 

\subsection{Client requirements}

The original set of categories that we were asked to identify were the following.

\begin{itemize}
  \item Social Networking
  \item Email
  \item Sport
  \item News
  \item Shopping
  \item Weather
  \item Travel
  \item Maps
  \item Photos
  \item Instant Messaging
  \item Entertainment
  \item Financial
  \item Adult
  \item Other
\end{itemize}

Before going any further and just looking at the categories we were asked to classify, we can observe that some of them could be more difficult to classify.
Some categories like "Adult" had a clearly characteristic language in their content whereas other categories like "Email" might contain generic language.
We can also see that the category "Entertainment" might be overlapping with others like "Photos", "Travel", "Social Networking", etc. 
Finally one of the difficulties of the task was to categorize something as "Other". 
For initial testing purposes that could help taking later decisions I used rapidminer. A fast-to-set and easy-to-use machine learning tool that, among other things, allowed me to test different
machine learning algorithms with the same datasets in order to compare their performance.

\subsection{Crawling training data}
Supervised Machine Learning algorithms need a set of training (labeled) data to infere the function that will later determine wether a new example is from one class or the other. In this case we needed
web content for each of the categories. We decided to use the {\bf Yahoo! Directory}\cite{yahoo} as a source for this data.
 

  


\subsection{Tests with Rapidminer}
I used rapidminer\cite{rapidminer} to do inicial tests and try to diagnose how might some categories be overlapping and at the same time detecting which algorithms performed better.
\imatgePetita{img/rapid1.png}{Rapidminer process}


\subsection{MaxEnt implementation in the Framework}

\[
p = \frac{e^{p(i)}}{e^{\sum_{i=0}^{n} e^{i-max}}}
\]

