%!TEX root = main.tex

The quick growth in the telecommunication industry is letting millions of people use their mobile devices to connect to the internet and communicate in a different way. The post-PC era is here and
the amount of consumed data per second is increasing together with the information that users themselves generate simply by interacting in the network.\\
Although the classical mobile device usage (phone calls) is still growing linearly the modern way of using them (internet, social networks, video, etc.) has been growing exponentially since the year 
2007 \ref{fig:{img/ericsson_global.png}}.
\imatge{img/ericsson_global.png}{Total monthly mobile traffic in 2011} \\
In order to squeeze all the information users are generating, and to be able to provide a better customer experience, mobile internet service providers need new powerful tools that are able to precisely
measure important variables in the key elements of their infrastructures. Using these tools, providers can focus on their tasks while measuring and improving their quality of service.
Zhilabs is one of the companies that is building these kind of tools for telecommunication operators with their star product called FlowSight\cite{flowsight} and this thesis is the result of the 
extension of the FlowSight platform capabilities after one of their client's request.  \\ 
The remainder of the document is organized as follows. In Chapter \ref{chap:flowsight} I introduce the FlowSight platform describing its characteristics and functionalities, and explaining my
expected contribution to it. In Chapter \ref{chap:research} I analyze the client requirements and perform some initial tests with a subset of the data. In Chapter \ref{chap:implementation} I introduce
the concept of Maximum Entropy and focus in the implementation and performance evaluation of the algorithm. In Chapter \ref{chap:enhancements} I discuss the obtained results and as well as ways of 
improving the quality and accuracy in the categorization. Finally, I describe some conclusions and future work in Chapter \ref{chap:conclusions}.
