%!TEX root = main.tex

%For each of the websites the module had to classify, information about the volume of consumed data on each of the connections was also available. 
%Figure \ref{fig:{img/volume_stacked.png}} is the result of combining the classification results with that information, before applying the language classification.
%the other data,

As we saw in previous chapters, FlowSight was already able to graphically represent the amount of consumed data in time \ref{fig:{img/total_traffic_browsing.png}}. The objective of this thesis was to
give FlowSight the power to accurately and efficiently categorize the traffic depending on the content of the visited URL. As we can see in figures \ref{fig:{img/volume_stacked.png}} and 
\ref{fig:{img/volume_plain.png}} the objective was achieved.

Looking at the results we can see how the consumption of URL of each category varies over time of the day, being entertainment and social networking the most volume consumer categories during the day and lowering their volume during the night, when adult category content becomes the top consumer.
Having this kind of information mobile internet service providers might be able to improve their balancing, giving more bandwith to those sites that might be more demanding at a given time of the day depending on their category. 

\imatge{img/total_traffic_browsing.png}{Total browsing traffic}
\imatge{img/volume_stacked.png}{Stacked representation of the volume of different categories}
\imatge{img/volume_plain.png}{Plain representation of the volume of different categories}
