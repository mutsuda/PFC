\documentclass[12pt, a4paper , titlepage]{report}
\usepackage{latexmasu}
\usepackage{fancyhdr}
\setcounter{tocdepth}{4}
\pagestyle{fancy}
\lhead{Automatic web content categorization}
\chead{} 
\rhead{\nouppercase{\leftmark}}
% \lfoot{\today}
% \cfoot{}
% \rfoot{\thepage}
\renewcommand{\headrulewidth}{0.4pt}
% \renewcommand{\footrulewidth}{0.4pt}

\renewcommand{\chaptermark}[1]{%
\markboth{\chaptername.  \thechapter. #1}{}}

\setlength{\parindent}{0pt}
\setlength{\parskip}{1ex plus 0.5ex minus 0.2ex}

\author{Masumi Mutsuda Zapater}
\title{Automatic web content categorization \\ Machine Learning \\ \large{\textit{Previous report}}}
\date{February 2012}

\begin{document}

\maketitle

\tableofcontents


\chapter{Overall objectives}
The main objectives of this thesis are:
- Implementing and configuring a machine learning algorithm able to categorize web content automatically inside FlowSight. 
It will serve as an enchancement of a larger application called FlowSight that analizes subscriber generated internet traffic within the network of different telecom companies.
- Initial tests for determining which algorithms perform better
- Solve language problems (lid) for better accuracy
- Analyzing the results as well as the algorithm performance is also part of the project.


\chapter{Covered objectives}
The enchancement provided by this thesis has been already used by the company, in some of their projects. 


\chapter{Planification}




\end{document}
