\documentclass[12pt, a4paper , titlepage]{report}
\usepackage{latexmasu}
\usepackage{fancyhdr}
\usepackage{colortbl}
\setcounter{tocdepth}{4}
\pagestyle{fancy}
\lhead{Automatic web content categorization}
\chead{} 
\rhead{\nouppercase{\leftmark}}
% \lfoot{\today}
% \cfoot{}
% \rfoot{\thepage}
\renewcommand{\headrulewidth}{0.4pt}
% \renewcommand{\footrulewidth}{0.4pt}

\renewcommand{\chaptermark}[1]{%
\markboth{\chaptername.  \thechapter. #1}{}}

\setlength{\parindent}{0pt}
\setlength{\parskip}{1ex plus 0.5ex minus 0.2ex}

\author{Masumi Mutsuda Zapater}
\title{Automatic web content categorization \\ Machine Learning \\ \large{\textit{Previous report}}}
\date{February 2012}

\begin{document}

\maketitle

\tableofcontents

\chapter{Introduction}
FlowSight is a Massively Parallel Processing platform that scales using Commercial Off-The-Shelf Hardware and Directly Attached Storage. It has been conceived among other things, to collect
huge amounts of data from the mobile internet networks analyzing it in real time, correlating it and analyzing the results in order to act upon. FlowSight includes:
\begin{itemize}
  \item{State of the art Fault Tolerance, Distribution and Manageability features that allow for low response latencies and always-on Analytics.}
  \item{A next generation Complex Event Processing engine that is able to achieve processing rates that were previously only available at packet processing level. The engine uses schema-less, 
        distributed, in-memory columnar storage to be able to handle massive amounts of data.}
  \item{Built-in reporting capabilities}
\end{itemize}
The figure\ref{fig:{img/total_traffic_type.png}} represents three days of data capturing and analyzing. As we can see the application is already able to distinguish between differents types of traffic (OTT, streaming, browsing...).
But looking at the browsing category\ref{fig:{img/total_traffic_browsing.png}}, all we know right now is the amount of data that is consumed and when it is consumed.

\imatge{img/total_traffic_type.png}{Total volume per traffic type}

\imatge{img/total_traffic_browsing.png}{Total browsing volume}

The main objective of this thesis, is to be able to split the browsing category into subcategories by automatically (with Machine Learning algorithms) categorizing the content of each web page that is browsed. 


\chapter{Overall objectives}
This thesis is centered in the extension of the FlowSight capabilities. In this case, the need for improving FlowSight was driven by a client request and the schedule for its completion was tightly
adjusted. As we already saw in the previous section, the main goal of this thesis is to implement and configure a module in the already existing platform FlowSight, to be able to differentiate
the content of the websites that are being visited.\\
 


\chapter{Covered objectives}

The following table represents the objectives of the thesis which have already been covered.

\begin{center}  
    \begin{tabular}{ | p{10cm} | l |}
    \hline
    \rowcolor{lightgray}{\bf Objective} & {\bf Date} \\ \hline
    Learning to use the FlowSight platform & September 2011 \\ \hline 
    Studying the implementation of the FlowSight platform  & September 2011 \\ \hline
    Analyzing the client requirements & October 2011 \\ \hline
    Tuning the existing code to match the client requirements & October 2011 \\ \hline
    Crawling the Yahoo! directory to obtain a reliable corpus & October 2011\\ \hline
    Performing initial tests with Rapidminer & November 2011 \\ \hline
    Implementation and setup in the Framework & November 2011\\ \hline
    Enhancements & December 2011\\ \hline
    \end{tabular}
\end{center}

\chapter{Plan}
In the following table the pending tasks are enumerated as well as an approximated date of their completion.
\begin{center}
    \begin{tabular}{ | p{10cm} | l |}
    \hline
    \rowcolor{lightgray}{\bf Objective} & {\bf Date} \\ \hline
    Performance evaluation & Mid April 2012\\ \hline
    Result analysis & Late April 2012\\ \hline
    Conclusions and future work & May 2012\\ \hline
    Documentation and presentation & Late May\\ \hline
    \end{tabular}
\end{center}
 



\end{document}
